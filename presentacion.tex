%%%
% Plantilla de Presentación
% Modificación de una plantilla de Latex de LaTeXTemplates para adaptarla 
% al castellano y a las necesidades de escribir informática y matemáticas.
%
% Editada por: Mario Román
%
% License:
% CC BY-NC-SA 3.0 (http://creativecommons.org/licenses/by-nc-sa/3.0/)
%%%

%%%%%%%%%%%%%%%%%%%%%%%%%%%%%%%%%%%%%%%%%
% Beamer Presentation
% LaTeX Template
% Version 1.0 (10/11/12)
%
% This template has been downloaded from:
% http://www.LaTeXTemplates.com
%
% License:
% CC BY-NC-SA 3.0 (http://creativecommons.org/licenses/by-nc-sa/3.0/)
%
%%%%%%%%%%%%%%%%%%%%%%%%%%%%%%%%%%%%%%%%%

%----------------------------------------------------------------------------------------
%	PAQUETES Y CONFIGURACIÓN DEL DOCUMENTO
%----------------------------------------------------------------------------------------

\documentclass{beamer}

%% Configuración de la presentación
\mode<presentation> {
  %%% Selección de estilo
  % The Beamer class comes with a number of default slide themes
  % which change the colors and layouts of slides. Below this is a list
  % of all the themes, uncomment each in turn to see what they look like.

  %\usetheme{default}
  %\usetheme{AnnArbor}
  %\usetheme{Antibes}
  %\usetheme{Bergen}
  %\usetheme{Berkeley}
  %\usetheme{Berlin}
  %\usetheme{Boadilla}
  %\usetheme{CambridgeUS}
  %\usetheme{Copenhagen}
  %\usetheme{Darmstadt}
  %\usetheme{Dresden}
  %\usetheme{Frankfurt}
  %\usetheme{Goettingen}
  %\usetheme{Hannover}
  %\usetheme{Ilmenau}
  %\usetheme{JuanLesPins}
  %\usetheme{Luebeck}
  \usetheme{Madrid}
  %\usetheme{Malmoe}
  %\usetheme{Marburg}
  %\usetheme{Montpellier}
  %\usetheme{PaloAlto}
  %\usetheme{Pittsburgh}
  %\usetheme{Rochester}
  %\usetheme{Singapore}
  %\usetheme{Szeged}
  %\usetheme{Warsaw}

  %% Selección de color
  % As well as themes, the Beamer class has a number of color themes
  % for any slide theme. Uncomment each of these in turn to see how it
  % changes the colors of your current slide theme.

  %\usecolortheme{albatross}
  %\usecolortheme{beaver}
  %\usecolortheme{beetle}
  %\usecolortheme{crane}
  %\usecolortheme{dolphin}
  %\usecolortheme{dove}
  %\usecolortheme{fly}
  %\usecolortheme{lily}
  %\usecolortheme{orchid}
  %\usecolortheme{rose}
  %\usecolortheme{seagull}
  %\usecolortheme{seahorse}
  %\usecolortheme{whale}
  %\usecolortheme{wolverine}

  %% Configuración del pie de línea
  %\setbeamertemplate{footline} % To remove the footer line in all slides uncomment this line
  %\setbeamertemplate{footline}[page number] % To replace the footer line in all slides with a simple slide count uncomment this line
  %\setbeamertemplate{navigation symbols}{} % To remove the navigation symbols from the bottom of all slides uncomment this line
}

%% Fuentes de tamaño arbitrario
\usepackage{lmodern}

%% Gráficos
\usepackage{graphicx} % Allows including images
\usepackage{booktabs} % Allows the use of \toprule, \midrule and \bottomrule in tables

%%% Castellano.
% noquoting: Permite uso de comillas no españolas.
% lcroman: Permite la enumeración con numerales romanos en minúscula.
% fontenc: Usa la fuente completa para que pueda copiarse correctamente del pdf.
\usepackage[spanish,es-noquoting,es-lcroman]{babel}
\usepackage[utf8]{inputenc}
\usepackage[T1]{fontenc}
\selectlanguage{spanish}

%----------------------------------------------------------------------------------------
%	TÍTULO
%----------------------------------------------------------------------------------------

\title[Título]{Protocolo SIP y VoIP} % The short title appears at the bottom of every slide, the full title is only on the title page

\author{Óscar Bermúdez, Lothar Soto} % Your name
\institute[UGR] % Your institution as it will appear on the bottom of every slide, may be shorthand to save space
{
  Universidad de Granada \\ % Your institution for the title page
  \medskip
  \textit{autor@ugr.correo.es} % Your email address
}
\date{\today} % Date, can be changed to a custom date



\begin{document}

%% Diapositiva de título.
\begin{frame}
\titlepage % Print the title page as the first slide
\end{frame}

%% Diapositiva de contenidos.
% Throughout your presentation, if you choose to use \section{} and \subsection{} commands, 
% these will automatically be printed on this slide as an overview of your presentation
\begin{frame}
  \frametitle{Contenidos} % Table of contents slide, comment this block out to remove it
  \tableofcontents
\end{frame}



%----------------------------------------------------------------------------------------
%	PRESENTACIÓN
%----------------------------------------------------------------------------------------

%------------------------------------------------
\section{SIP} % Sections can be created in order to organize your presentation into discrete blocks, all sections and subsections are automatically printed in the table of contents as an overview of the talk
%------------------------------------------------

\subsection{Un poco de historia} % A subsection can be created just before a set of slides with a common theme to further break down your presentation into chunks
\begin{frame}
\frametitle{Un poco de historia}
SIP o Protocolo de Inicio de Sesiones es un protocolo desarrollado por el \textbf{IETF MMUSIC Working Group} con la intención de ser el estándar para la iniciación, modificación y finalización de sesiones interactivas de usuario donde intervienen elementos multimedia.
	
	Puede funcionar en Transmission Control Protocol(\textbf{TCP}), User Datagram Protocol(\textbf{UDP}) o Stream Control Transmission Protocol(\textbf{SCTP}).\\~\\
	Fue publicado en:
	\begin{itemize}
	\item RFC\_2543, RFC\_3261.
	\end{itemize}
\end{frame}
\subsection{Funcionalidades}
\begin{frame}
\frametitle{Especificación, Modificación y fin de sesión}
El estandar SIP define la forma en la que se lleva a cabo el establecimiento, modificación y el fin de comunicación multimedia. \\~\\
Para iniciar un proceso de comunicación es necesario que se que:
		\begin{itemize}
			\item El usuario añadido al proceso debe aceptar participar en dicha sesión.
			\item Los usuarios deben establecer los parámetros multimedia a utilizar.\\~\\
		\end{itemize}	
\end{frame}
\begin{frame}
\frametitle{Funcionalidad}
		El funcionamiento de SIP se realiza de la siguiente forma:
		\begin{itemize}
			\item Los usuarios establecen los códec de voz y video a usar u otros parámetros multimedia.
			\item Si se producen cambios durante la comunicación se notifican a los usuarios que formen parte de la misma.
			\item Por último en el momento en el que uno de los usuarios desea llevar a cabo la desconexión, se notifica al resto de usuarios de la misma.
		\end{itemize}
\end{frame}
\begin{frame}
\frametitle{Movilidad}
		Una de las más importantes ventajas de la telefonia IP es que se puede usar el servicio sin necesidad de encontrarse en una red especifia, el protocolo SIP antes de realizar la comunicación entre usuarios requiere del reconocimiento de la dirección IP.\\~\\
	Herramientas que usa:
		\begin{description}
		\item[-]\textbf{URL SIP:} Se le asigna una URL a cada usuario de la red con el objetivo de dar una referencia única en internet. Tiene el siguiente formato:\\
		\begin{example}[URL SIP:]
		SIP://<user>:<password>@<host><tlf>:<PORT>
		\end{example}

		\item[-]\textbf{Registros:} Esto permite al usuario cambiar su ubicación en lo que a dirección IP.
		\end{description}
\end{frame}
\subsection{Arquitectura}
\begin{frame}
\frametitle{Peticiones}
		\begin{itemize}
			\item \textbf{REGISTER}: Usado por un agente de usuario para registrarse.
			\item \textbf{INVITE}: Usado para establecer una sesión multimedia entre agentes de usuario.
			\item \textbf{ACK}: Confirma que los intercambios de mensajes son confiables.
			\item \textbf{CANCEL}: Cancela una petición pendiente.
			\item \textbf{BYE}: Termina una sesión.
			\item \textbf{OPTIONS}: Solicita información sobre las posibilidades de un agente de usuario sin la necesidad de comenzar una sesión.
		\end{itemize}
\end{frame}
\begin{frame}
\frametitle{Respuestas}
\begin{itemize}
			\item \textbf{Provisional(1xx)}: La petición fue recibida y está siendo  procesada.
			\item \textbf{Success(2xx)}: La acción fue recibida satisfactoriamente, entendida y aceptada.
			\item \textbf{Redirection(3xx)}: Nuevas acciones necesitan ser realizadas para completar la petición.
			\item \textbf{Client Error(4xx)}: La petición tiene un fallo de sintaxis o no puede ser rellenada por el servidor.
			\item \textbf{Server Error(5xx)}: El servidor falló al rellenar una petición aparentemente válida.
			\item \textbf{Global Failure(6xx)}: La petición no puede ser rellenada en ningún servidor.
		\end{itemize}
\end{frame}
\subsection{Aplicaciones SIP}
\begin{frame}
\frametitle{Aplicaciones SIP}
SIP define una arquitectura de señalización y control ampliamente para llamadas de voz y vídeo sobre Internet Protocol(VoIP). También puede ser usado para crear, modificar y terminar sesiones de streaming.\\~\\
Programas que usan SIP son:
		\textbf{\begin{itemize}
			\item Twinkle
			\item Tapioca
			\item KCall
		\end{itemize}}
\end{frame}
\begin{frame}
\frametitle{Paragraphs of Text}
Sed iaculis dapibus gravida. Morbi sed tortor erat, nec interdum arcu. Sed id lorem lectus. Quisque viverra augue id sem ornare non aliquam nibh tristique. Aenean in ligula nisl. Nulla sed tellus ipsum. Donec vestibulum ligula non lorem vulputate fermentum accumsan neque mollis.\\~\\

Sed diam enim, sagittis nec condimentum sit amet, ullamcorper sit amet libero. Aliquam vel dui orci, a porta odio. Nullam id suscipit ipsum. Aenean lobortis commodo sem, ut commodo leo gravida vitae. Pellentesque vehicula ante iaculis arcu pretium rutrum eget sit amet purus. Integer ornare nulla quis neque ultrices lobortis. Vestibulum ultrices tincidunt libero, quis commodo erat ullamcorper id.
\end{frame}

%------------------------------------------------

\begin{frame}
\frametitle{Bullet Points}
\begin{itemize}
\item Lorem ipsum dolor sit amet, consectetur adipiscing elit
\item Aliquam blandit faucibus nisi, sit amet dapibus enim tempus eu
\item Nulla commodo, erat quis gravida posuere, elit lacus lobortis est, quis porttitor odio mauris at libero
\item Nam cursus est eget velit posuere pellentesque
\item Vestibulum faucibus velit a augue condimentum quis convallis nulla gravida
\end{itemize}
\end{frame}

%------------------------------------------------

\begin{frame}
\frametitle{Blocks of Highlighted Text}
\begin{block}{Block 1}
Lorem ipsum dolor sit amet, consectetur adipiscing elit. Integer lectus nisl, ultricies in feugiat rutrum, porttitor sit amet augue. Aliquam ut tortor mauris. Sed volutpat ante purus, quis accumsan dolor.
\end{block}

\begin{block}{Block 2}
Pellentesque sed tellus purus. Class aptent taciti sociosqu ad litora torquent per conubia nostra, per inceptos himenaeos. Vestibulum quis magna at risus dictum tempor eu vitae velit.
\end{block}

\begin{block}{Block 3}
Suspendisse tincidunt sagittis gravida. Curabitur condimentum, enim sed venenatis rutrum, ipsum neque consectetur orci, sed blandit justo nisi ac lacus.
\end{block}
\end{frame}

%------------------------------------------------

\begin{frame}
\frametitle{Multiple Columns}
\begin{columns}[c] % The "c" option specifies centered vertical alignment while the "t" option is used for top vertical alignment

\column{.45\textwidth} % Left column and width
\textbf{Heading}
\begin{enumerate}
\item Statement
\item Explanation
\item Example
\end{enumerate}

\column{.5\textwidth} % Right column and width
Lorem ipsum dolor sit amet, consectetur adipiscing elit. Integer lectus nisl, ultricies in feugiat rutrum, porttitor sit amet augue. Aliquam ut tortor mauris. Sed volutpat ante purus, quis accumsan dolor.

\end{columns}
\end{frame}

%------------------------------------------------
\section{VoIP}
%------------------------------------------------
\subsection{Telefonia IP vs Telefonia convencional}
\subsection{Calidad de servicio (QoS)}
\subsection{Requisitos QoS}

\begin{frame}
\frametitle{Table}
\begin{table}
\begin{tabular}{l l l}
\toprule
\textbf{Treatments} & \textbf{Response 1} & \textbf{Response 2}\\
\midrule
Treatment 1 & 0.0003262 & 0.562 \\
Treatment 2 & 0.0015681 & 0.910 \\
Treatment 3 & 0.0009271 & 0.296 \\
\bottomrule
\end{tabular}
\caption{Table caption}
\end{table}
\end{frame}

%------------------------------------------------

\begin{frame}
\frametitle{Theorem}
\begin{theorem}[Mass--energy equivalence]
$E = mc^2$
\end{theorem}
\end{frame}

%------------------------------------------------

\begin{frame}[fragile] % Need to use the fragile option when verbatim is used in the slide
\frametitle{Verbatim}
\begin{example}[Theorem Slide Code]
\begin{verbatim}
\begin{frame}
\frametitle{Theorem}
\begin{theorem}[Mass--energy equivalence]
$E = mc^2$
\end{theorem}
\end{frame}\end{verbatim}
\end{example}
\end{frame}

%------------------------------------------------

\begin{frame}
\frametitle{Figure}
Uncomment the code on this slide to include your own image from the same directory as the template .TeX file.
%\begin{figure}
%\includegraphics[width=0.8\linewidth]{test}
%\end{figure}
\end{frame}

%------------------------------------------------

\begin{frame}[fragile] % Need to use the fragile option when verbatim is used in the slide
\frametitle{Citation}
An example of the \verb|\cite| command to cite within the presentation:\\~

This statement requires citation \cite{p1}.
\end{frame}

%------------------------------------------------

%% Bibliografía
\begin{frame}
\frametitle{Referencias}
\footnotesize{
  \begin{thebibliography}{99} % Beamer does not support BibTeX so references must be inserted manually as below
    \bibitem[Smith, 2012]{p1} John Smith (2012)
      \newblock Title of the publication
      \newblock \emph{Journal Name} 12(3), 45 -- 678.
  \end{thebibliography}
}
\end{frame}

%------------------------------------------------

\begin{frame}
\Huge{\centerline{Fin.}}
\end{frame}

%----------------------------------------------------------------------------------------

\end{document} 